\documentclass[12pt]{article}

\usepackage[utf8]{inputenc}
\usepackage[english,russian]{babel}
\usepackage[T2A]{fontenc}
\usepackage{textcase}
\usepackage{titlesec}
\usepackage{needspace}
\usepackage{float}
\usepackage{graphicx}
\usepackage{geometry}
\geometry{a4paper,tmargin=1in,bmargin=1in,lmargin=2.5cm,rmargin=2cm}

\titleformat{\section}
{\needspace{1em}\normalfont\Large\bfseries}{\thesection}{1em}{}
\titleformat{\subsection}
{\needspace{1em}\normalfont}{\thesubsection}{1em}{}
\titleformat{\subsubsection}
{\needspace{1em}\normalfont}{\thesubsubsection}{1em}{}

\begin{document}

\begin{center}
{\Large \MakeUppercase{Договор авторского заказа} \\
(с отчуждением исключительных прав)}
\end{center}

\vspace{.5cm}

\noindent <<\underline{\hspace{0.75cm}}>> \underline{\hspace{3cm}} 20\underline{\hspace{0.75cm}} года \textnumero 1/20\underline{\hspace{0.75cm}} г.  \underline{\hspace{3cm}}

\vspace{12pt}
\hrule height .1mm\vspace{12pt}
\hrule height .1mm\vspace{12pt}
\hrule height .1mm\vspace{12pt}

именуемый в дальнейшем <<Заказчик>>, с одной стороны и
гражданин Российской Федерации

\vspace{12pt}
\hrule height .1mm\vspace{12pt}
\hrule height .1mm\vspace{12pt}
\hrule height .1mm\vspace{12pt}

именуемый в дальнейшем <<Автор>> с другой Стороны, далее именуемые <<Стороны>>, а по отдельности --- <<Сторона>>,
заключили настоящий Договор авторского заказа с отчуждением исключительных прав (далее --- <<Договор>>) о нижеследующем:

\section*{Основные понятия Договора}

Автор --- физическое лицо (гражданин), творческим трудом которого создано произведение (статья 1228 ГК РФ).

Задание --- письменное требование Заказчика к результатам интеллектуальной деятельности Автора, созданным по настоящему договору (Приложение \textnumero 1 к настоящему Договору).

Интеллектуальная деятельность --- созидательная умственная творческая деятельность человека для достижения каких-либо результатов.

Интеллектуальная собственность --- результаты интеллектуальной деятельности и приравненные к ним средства индивидуализации, которым предоставляется правовая охрана законодательством РФ (статья 1225 ГК РФ).

Использование --- воспроизведение, распространение, публичный показ произведения, импорт оригинала или экземпляра произведения, а также иное его использование, предусмотренное законом или договором (статья 1270 ГК РФ).

Конфиденциальная информация --- любая информация технического, коммерческого, финансового характера, прямо или косвенно относящаяся к настоящему Договору (непосредственно Произведения, информация о будущих публикациях, о содержании неопубликованных Произведений, стратегия дальнейшей работы и т.п.), а также к отношениям деятельности Сторон и их партнёров, ставшая известной сторонам в ходе выполнения настоящего Договора или предварительных переговоров о его заключении, либо отнесённая законом к конфиденциальной.

Обнародование --- осуществленное с согласия Автора действие, которое впервые делает произведение доступным для всеобщего сведения путем его опубликования, публичного показа, публичного исполнения, сообщения в эфир или по кабелю любо иным способом (статья 1268 ГК РФ).

Опубликование --- выпуск в обращение экземпляров, представляющих собой копию произведения в любой материальной форме (на материальном носителе), в количестве, достаточным для удовлетворения разумных потребностей публики исходя из характера произведения (статья 1268 ГК РФ).

Правообладатель --- гражданин или юридическое лицо, обладающее исключительным правом на интеллектуальную собственность (статья 1229 ГК РФ).

Произведение --- результат интеллектуальной деятельности (произведения науки, литературы и искусства, программы для ЭВМ и базы данных), охраняемый авторским правом, о создании которого заключен настоящий Договор.

Результат интеллектуальной деятельности --- продукт умственной творческой деятельности человека по перечню согласно статье 1225 ГК РФ, созданный в порядке служебного задания, подряда, аутсорсинга или личной инициативы и являющийся неотчуждаемым объектом гражданских прав.

\section{Предмет Договора}

\subsection{Автор обязуется по Заданиям Заказчика создать Произведения и передать Заказчику исключительные права на Произведения, а Заказчик обязуется выплатить Автору вознаграждение в размере, порядке и на условиях, установленных настоящим Договором.}
\subsection{Произведения, создаваемые творческим трудом Автора, должны соответствовать требованиям Заказчика, содержащимся в Задании (Приложение \textnumero 1) к настоящему Договору.}

\section{Права и обязанности Сторон}

\subsection{Автор обязуется:}

\subsubsection{создать Произведения в соответствии с требованиями Заказчика, выраженными в соответствующих Заданиях;}

\subsubsection{личным творческим трудом, без привлечения третьих лиц, создать оригинальные Произведения, без нарушения прав третьих лиц, заимствований и плагиата;}

\subsubsection{в случае создания производного Произведения с использованием интеллектуальной собственности третьих лиц своими силами обеспечить соблюдение прав третьих лиц и за свой счёт (в случае необходимости) выплатить авторские вознаграждения;}

\subsubsection{создать Произведения в сроки, установленные Договором;}

\subsubsection{незамедлительно сообщать Заказчику о невозможности выполнения Задания к определенному в Задании сроку или в точном соответствии с требованиями Задания;}

\subsubsection{по запросу Заказчика информировать его о ходе работы над Произведением в форме и в сроки, определенные в Задании;}

\subsubsection{предварительно согласовывать с Заказчиком использование при создании Произведений ранее созданных Автором для третьих лиц произведений или их элементов;}

\subsubsection{передать Заказчику исключительные права на Произведения в полном объеме;}

\subsubsection{передать в собственность Заказчика материальные носители, в которых выражены Произведения, в виде и порядке, установленных в Задании.}

\subsection{Автор вправе:}

\subsubsection{запрашивать у Заказчика информацию, необходимую для выполнения Задания;}

\subsubsection{требовать выплаты авторского вознаграждения;}

\subsubsection{требовать увеличения сроков исполнения Договора и/или увеличения размера авторского вознаграждения в случае изменения (корректировки) Заказчиком Задания после заключения Договора, либо отказаться от исполнения Договора в одностороннем порядке, при этом Заказчик должен выплатить Автору вознаграждение за созданные промежуточные результаты, а Автор передать права на данные результаты в соответствии с п.6.6. настоящего Договора.}

\subsection{Заказчик обязуется:}

\subsubsection{передавать автору Задание и изменения в Задании (в случае возникновения таковых) в письменном виде;}

\subsubsection{предоставить Автору всю информацию, необходимую для исполнения Договора по Акту приёма-передачи исходных материалов (Приложение \textnumero 2 к настоящему Договору);}

\subsubsection{выплатить Автору вознаграждение в размере, порядке и на условиях, предусмотренных настоящим Договором.}

\subsection{Заказчик вправе:}

\subsubsection{проверять надлежащее выполнение Автором своих обязательств по настоящему Договору;}

\subsubsection{корректировать Задания в ходе их исполнения, предварительно согласовав такие изменения с Автором;}

\subsubsection{в любое время требовать предоставления Автором промежуточных и окончательных итогов выполнения Заданий;}

\subsubsection{при обнаружении в Произведениях несоответствия Заданиям или любых нарушений Договора требовать их устранения или переделки;}

\subsubsection{отказаться от оплаты вознаграждения и требовать возврата аванса в случае неисполнения или ненадлежащего исполнения Автором своих обязательств по настоящему Договору;}

\subsubsection{после выплаты вознаграждения использовать Произведения по собственному усмотрению в любой форме и любыми не противоречащими законодательству и настоящему Договору способами.}

\newpage
\section{Переход исключительных прав}

\subsection{Все отчуждаемые исключительные права на Произведения, созданные Автором в рамках выполнения настоящего Договора, а равно на составные части, элементы таких произведений, а также иные права на результаты исполнения Договора, которые не являются охраняемыми результатами интеллектуальной деятельности, принадлежат Заказчику.}

\subsection{Заказчик приобретает исключительные права на Произведение и иные объекты интеллектуальной собственности, созданные по настоящему Договору, включая составные и производные произведения, являющиеся самостоятельными объектами авторского права, части произведения, которые имеют самостоятельное значение, иные объекты интеллектуальной собственности.}

\subsection{Момент перехода к Заказчику прав на результаты интеллектуальной деятельности, возникшие в результате исполнения настоящего Договора, как по отдельным этапам его выполнения, так и по Договору в целом, наступает в день подписания Акта об исполнении настоящего Договора (Приложение \textnumero 3 к настоящему Договору). Неотъемлемыми приложениями указанного Акта являются надлежаще оформленные документы, подтверждающие наличие и использование интеллектуальной собственности без нарушения интеллектуальных прав.}

\subsection{Исключительные права на Произведения передаются Автором Заказчику в полном объеме.}

\subsection{Автор оставляет за собой право использовать Произведения и их материальные носители для демонстрации своим потенциальным клиентам в качестве образцов, в том числе путем размещения на своем официальном сайте, а также указывать наименование Заказчика данных Произведений.}

\subsection{Заказчик по своему усмотрению может осуществлять воспроизведение, распространение, обнародование, сообщение в эфир и по кабелю, переработку, перевод Произведений, а также использовать Произведения любым не противоречащим закону способом, только при условии обязательного указания имени, псевдонима или другого знака Автора в согласованном Сторонами виде.}

\subsection*{\bf Территория отчуждаемых прав}

\subsection{Территория действия отчуждаемого в пользу Заказчика исключительного права не ограничена (весь мир).}

\subsection*{\bf Срок действия прав}

\subsection{Автор не устанавливает самостоятельно ограничений на срок действия комплекса исключительных прав, передаваемых по настоящему Договору.}

\subsection{Срок действия комплекса исключительных прав, передаваемых по настоящему Договору равняется всему сроку действия данных прав, в соответствии с законодательством Российской Федерации.}

\subsection{В случае, если срок действия исключительного права впоследствии будет изменен законодательством Российской Федерации или международными актами, признаваемыми и применяемыми в Российской Федерации, срок действия исключительного права может быть изменен в сторону увеличения или уменьшения, если это будет прямо предусмотрено соответствующими изменениями.}

\subsection*{\bf Гарантии и заверения}

\subsection{Автор гарантирует, что является единственным правообладателем исключительных прав на Произведения и не передаст исключительные права третьим лицам.}

\subsection{Автор гарантирует, что Произведения будут созданы исключительно его творческим трудом без привлечения соавторов, третьих лиц, а также без заимствований, плагиата и неправомерного цитирования других произведений, в том числе общеизвестных и общедоступных; и что при создании Произведений не будут использоваться результаты интеллектуальной деятельности, созданные или создаваемые для Заказчика и других Заказчиков. Использование объектов интеллектуальной собственности третьих лиц при создании Произведений возможно только с письменного согласования Заказчика и при предоставлении Заказчику прав и подтверждающих документов на использование интеллектуальной собственности третьих лиц на законных основаниях без нарушения прав.}

\subsection{В целях идентификации Произведения к Акту по п.3.3 Договора (Приложение \textnumero 3 к настоящему Договору) прилагаются:}

\subsubsection{по одному экземпляру каждого Произведения с собственноручной подписью и знаком охраны авторского права Автора, помещенным на экземпляре Произведения в соответствии с требованиями ст. 1271 Гражданского кодекса Российской Федерации;}

\subsubsection{описание основных характеристик каждого Произведения (указываются: жанр, тематика, отличительные особенности произведения, особенности построения и назначение, иные сведения, позволяющие идентифицировать Произведения);}

\subsection{После передачи интеллектуальных прав Заказчику Автор гарантирует, что не будет передавать материальные носители, в которых выражены Произведения и исключительные права на Произведения третьим лицам, не будет публиковать Произведения или их части, в том числе в любых других изданиях, на веб-страницах в любом виде, не предусмотренном настоящим Договором.}

\section{Вознаграждение и порядок оплаты}

\subsection{За создание Произведений и передачу исключительных прав на них Заказчик выплачивает Автору вознаграждение, размер которого определен в Приложении \textnumero 1 к настоящему Договору. Размер вознаграждения устанавливается для каждого Задания и каждого Произведения отдельно.}

\subsection{Вознаграждение выплачивается Заказчиком в следующем порядке:}

\subsubsection{Аванс в размере \underline{\hspace{0.75cm}}\% (\underline{\hspace{4cm}} процентов) от общей суммы вознаграждения Заказчик выплачивает в течение 5 (пяти) рабочих дней с даты подписания настоящего Договора.}

\subsubsection{Остальную часть (части) вознаграждения Заказчик выплачивает в течение 5 (пяти) рабочих дней с даты подписания Сторонами Акта об исполнении настоящего Договора или отдельного этапа (этапов) настоящего Договора (Приложение \textnumero 3 к настоящему Договору), подтверждающего создание Произведений и передачу исключительных прав на них Заказчику в соответствии с условиями настоящего Договора.}

\subsection{Вознаграждение по настоящему Договору включает в себя:}

\subsubsection{цену работ по созданию Произведений и стоимость материальных носителей экземпляров Произведений, передаваемых в собственность Заказчику;}

\subsubsection{вознаграждение за отчуждение исключительных прав на Произведения, созданные Автором;}

\subsubsection{любые расходы Автора, связанные с выполнением Задания.}

\subsection{За досрочное создание каждого Произведения дополнительное Вознаграждение Автору не выплачивается.}

\subsection{Получение Автором авторского вознаграждения по настоящему Договору является исчерпывающей реализацией им своего права на получение каких-либо дополнительных выплат за использование Заказчиком Произведений, в том числе за использование производных результатов интеллектуальной деятельности, потенциально охранноспособных технических решений и любых других отдельных элементов Произведений. В случаях отчуждения Заказчиком прав на Произведение, производные объекты или отдельные элементы Произведения третьим лицам, а также в случае заключения Заказчиком лицензионных (сублицензионных) договоров о предоставлении прав использования Произведения, дополнительное вознаграждение Автору не выплачивается.}

\subsection{В случае выявления нарушения условий настоящего Договора вознаграждение подлежит возврату Заказчику в полном объеме, не позднее 10 (десяти) рабочих дней от календарной даты поступления Автору обоснованных претензий Заказчика.}

\subsection*{\bf Порядок оплаты, налоговые выплаты}

\subsection{Оплата производится в безналичном порядке. Обязательства Заказчика по оплате считаются исполненными на дату зачисления денежных средств на корреспондентский счёт банка Автора.}

\subsection{Автор самостоятельно исчисляет и уплачивает с сумм, полученных по настоящему Договору от Заказчика, налог на доходы физических лиц.}

\subsection{Авторское вознаграждение, полученное Автором по настоящему Договору, не является объектом обложения страховых взносов (п. 2 ст. 7 Федерального закона <<О страховых взносах в Пенсионный фонд Российской Федерации, Фонд социального страхования Российской Федерации, Федеральный фонд обязательного медицинского страхования>>).}

\section{Сроки исполнения Договора}

\subsection{Произведения, создание которых предусмотрено настоящим Договором, должны быть переданы Заказчику в сроки, указанные в Приложении \textnumero 1 к настоящему Договору.}

\subsection{При наличии уважительных причин для завершения создания Произведений Автору предоставляется дополнительный льготный срок (ст. 1289 Гражданского кодекса Российской Федерации).}

\subsection{Максимальная продолжительность дополнительного льготного срока по каждому Произведению устанавливается в $1/4$ (одну четверть) от срока, установленного для создания данного Произведения. Точная продолжительность дополнительных льготных сроков назначается и согласуется Заказчиком и Автором в письменном виде.}

\section{Ответственность Сторон и одностороннее расторжение Договора}

\subsection{За неисполнение или ненадлежащее исполнение обязательств по настоящему Договору Заказчик и Автор несут ответственность в соответствии с действующим законодательством Российской Федерации.}

\subsection{В случае нарушения Автором интеллектуальных (имущественных и неимущественных) прав третьих лиц ответственность и возмещение убытков, а равно иные негативные последствия таких нарушений возлагаются на Автора.}

\subsection{В случае несоответствия Произведений заявленному Заданию, наличия фактических и грамматических ошибок, несоответствия текста нормам литературного русского языка, наличию несогласованной рекламы, некорректной работы программы для ЭВМ и любых иных несоответствий Заданию, Заказчик вправе потребовать переделки Произведений за счёт Автора либо возложить на Автора расходы по устранению таких недостатков и доработке Произведения.}

\subsection{Если по истечении льготного срока, предоставленного Автору в соответствии с условиями настоящего Договора, Автор не передаст Произведения Заказчику, Заказчик вправе расторгнуть настоящий Договор в одностороннем внесудебном порядке без выплаты вознаграждения.}

\subsection{Заказчик вправе расторгнуть Договор в одностороннем внесудебном порядке при условии оплаты вознаграждения за фактически подготовленные и готовые к передаче Произведения или части Произведений.}

\subsection{Автор обязан по требованию Заказчика передать ему по акту оплаченные промежуточные проекты Произведения, включая предусмотренные настоящим Договором исключительные права на созданные к моменту расторжения настоящего Договора результаты интеллектуальной деятельности (Приложение \textnumero 3 к настоящему Договору).}

\section{Конфиденциальность}

\subsection{Автор обязуется сохранять в тайне конфиденциальную информацию, которая стала ему доступна в результате заключения настоящего Договора с Заказчиком.}

\subsection{Стороны обязуются не разглашать и не передавать конфиденциальные сведения (информацию) третьим лицам, а также не использовать их любым другим образом, кроме как для выполнения Заданий по настоящему Договору.}

\subsection{Стороны обязуются принимать все необходимые меры для сохранения в тайне конфиденциальной информации и иной информации, ставшей известной им вследствие исполнения обязательств по Договору.}

\subsection{Обязанность по соблюдению конфиденциальности бессрочна.}

%\begin{samepage}
\subsection{К Конфиденциальной информации не относится информация, которая:}

\begin{itemize}
\item уже известна или имеется у Стороны на момент ее передачи;
\item стала известна получающей Стороне после ее передачи, причем получающей Стороне неизвестно о нарушении обязательств по неразглашению конфиденциальной информации, предусмотренных настоящим Договором и она не участвовала в нарушении данных обязательств;
\item становится общеизвестной в момент передачи или после него и к которой доступ был представлен передающей Стороной третьим лицам без ограничений;
\item стала общеизвестной иным образом, не по вине или упущению и не в результате нарушения настоящего Договора;
\item была передана без обязательства о сохранении конфиденциальности, причем передача была явным образом санкционирована заранее в письменном виде передающей Стороной.
\end{itemize}

\section{Разрешение споров}
%\end{samepage}

\subsection{Все споры и разногласия разрешаются Сторонами путём переговоров.}

\subsection{Неурегулированные в процессе переговоров споры разрешаются в суде в порядке, установленном действующим законодательством Российской Федерации, по месту нахождения Заказчика.}

\newpage
\section{Заключительные положения}

\subsection{Во всем остальном, что не предусмотрено настоящим Договором, Стороны руководствуются действующим законодательством Российской Федерации.}

\subsection{Любые изменения и дополнения к настоящему Договору действительны при условии, что они совершены в письменной форме и подписаны надлежаще уполномоченными на то представителями Сторон.}

\subsection{Обмен сообщениями и документами, связанными с исполнением настоящего Договора, осуществляется как правило в письменной форме, при этом допускается обмен электронными сообщениями при обязательном подтверждении получения сообщения адресатом.}

\subsection{Настоящий Договор вступает в силу с момента его подписания Сторонами и действует до полного исполнения обязательств обеими Сторонами.}

\subsection{Настоящий Договор составлен на русском языке в двух экземплярах, имеющих одинаковую юридическую силу, по одному экземпляру для каждой из Сторон.}

\subsection{Неотъемлемыми частями настоящего Договора являются следующие Приложения:}

\begin{itemize}
\item Задание на создание Произведений (Приложение \textnumero 1);
\item Акт приема-передачи исходных материалов (Приложение \textnumero 2);
\item Акт об исполнении Договора (Приложение \textnumero 3).
\end{itemize}

\newpage
\section{Адреса и реквизиты Сторон}

\begin{table}[h]
    \begin{tabular}{p{8cm}p{8cm}}
    \multicolumn{1}{c}{\bf Заказчик} & \multicolumn{1}{c}{\bf Разработчик} \\
    ~ & ~ \\
    % полное наименование заказчика
    ~ & 
    % полное наименование исполнителя
    ~ \\
    ~ & ~ \\
    Юридический адрес: & ~ \\
    % адрес и реквизиты заказчика
    ~ \\
    ИНН ~,\\
    КПП ~ \\
    р/с № ~ & 
    % адрес и реквизиты исполнителя
    ~ \\
    ~ & ~ \\
    % адрес и реквизиты банка заказчика
    ~ & 
    % адрес и реквизиты банка исполнителя
    ~ \\
    ~ & ~ \\
    % наименования должностей и подписи представителей заказчика и исполнителя
    ~ & ~ \\
    \underline{\hspace{4cm}} /  / & \underline{\hspace{4cm}} /  / \\
    \end{tabular}
\end{table}

\end{document}
